\documentclass[conference]{IEEEtran}
\IEEEoverridecommandlockouts
% The preceding line is only needed to identify funding in the first footnote. If that is unneeded, please comment it out.
\usepackage{cite}
\usepackage{amsmath,amssymb,amsfonts}
\usepackage{algorithmic}
\usepackage{textcomp}
\usepackage{xcolor}
\usepackage{graphicx,url}
\usepackage[utf8]{inputenc}
\usepackage[brazil]{babel}
%\usepackage[latin1]{inputenc}  
\usepackage{algpseudocode,algorithm}

\def\BibTeX{{\rm B\kern-.05em{\sc i\kern-.025em b}\kern-.08em
    T\kern-.1667em\lower.7ex\hbox{E}\kern-.125emX}}

\newcommand{\knn}{$k$NN}
\newcommand{\nome}{BT-\knn{}}

% Pacote para a definição de novas cores
\usepackage{xcolor}
% Definindo novas cores
\definecolor{verde}{rgb}{0.25,0.5,0.35}
\definecolor{jpurple}{rgb}{0.5,0,0.35}
\definecolor{darkgreen}{rgb}{0.0, 0.2, 0.13}
%\definecolor{oldmauve}{rgb}{0.4, 0.19, 0.28}
% Configurando layout para mostrar codigos Java
\usepackage{listings}

\newcommand{\estiloJava}{
\lstset{
    language=Java,
    basicstyle=\ttfamily\small,
    keywordstyle=\color{jpurple}\bfseries,
    stringstyle=\color{red},
    commentstyle=\color{verde},
    morecomment=[s][\color{blue}]{/**}{*/},
    extendedchars=true,
    showspaces=false,
    showstringspaces=false,
    numbers=left,
    numberstyle=\tiny,
    breaklines=true,
    backgroundcolor=\color{cyan!10},
    breakautoindent=true,
    captionpos=b,
    xleftmargin=0pt,
    tabsize=2
}}

\newcommand{\estiloR}{
  \lstset{ %
    language=R,                     % the language of the code
    basicstyle=\footnotesize,       % the size of the fonts that are used for the code
    numbers=left,                   % where to put the line-numbers
    numberstyle=\tiny\color{gray},  % the style that is used for the line-numbers
    stepnumber=1,                   % the step between two line-numbers. If it's 1, each line
                                    % will be numbered
    numbersep=5pt,                  % how far the line-numbers are from the code
    backgroundcolor=\color{white},  % choose the background color. You must add \usepackage{color}
    showspaces=false,               % show spaces adding particular underscores
    showstringspaces=false,         % underline spaces within strings
    showtabs=false,                 % show tabs within strings adding particular underscores
    frame=single,                   % adds a frame around the code
    rulecolor=\color{black},        % if not set, the frame-color may be changed on line-breaks within not-black text (e.g. commens (green here))
    tabsize=2,                      % sets default tabsize to 2 spaces
    captionpos=b,                   % sets the caption-position to bottom
    breaklines=true,                % sets automatic line breaking
    breakatwhitespace=false,        % sets if automatic breaks should only happen at whitespace
    title=\lstname,                 % show the filename of files included with \lstinputlisting;
                                    % also try caption instead of title
    keywordstyle=\color{blue},      % keyword style
    commentstyle=\color{darkgreen},   % comment style
    stringstyle=\color{red},      % string literal style
    escapeinside={\%*}{*)},         % if you want to add a comment within your code
    morekeywords={*,...}          % if you want to add more keywords to the set
}}

\renewcommand{\lstlistingname}{Listagem}

\lstset{ %
%  language=Octave,                % the language of the code
  basicstyle=\scriptsize,           % the size of the fonts that are used for the code
  frame=single,                   % adds a frame around the code
  breaklines=true,                % sets automatic line breaking
  breakatwhitespace=false,        % sets if automatic breaks should only happen at whitespace
  escapeinside={\%*}{*)},            % if you want to add LaTeX within your code
  extendedchars=true,
}  

%Pseudo-código
% Declaracoes em Português
\algrenewcommand\algorithmicend{\textbf{fim}}
\algrenewcommand\algorithmicdo{\textbf{faça}}
\algrenewcommand\algorithmicwhile{\textbf{enquanto}}
\algrenewcommand\algorithmicfor{\textbf{para}}
\algrenewcommand\algorithmicif{\textbf{se}}
\algrenewcommand\algorithmicthen{\textbf{então}}
\algrenewcommand\algorithmicelse{\textbf{senão}}
\algrenewcommand\algorithmicreturn{\textbf{devolve}}
\algrenewcommand\algorithmicfunction{\textbf{função}}

% Rearranja os finais de cada estrutura
\algrenewtext{EndWhile}{\algorithmicend\ \algorithmicwhile}
\algrenewtext{EndFor}{\algorithmicend\ \algorithmicfor}
\algrenewtext{EndIf}{\algorithmicend\ \algorithmicif}
\algrenewtext{EndFunction}{\algorithmicend\ \algorithmicfunction}

% O comando For, a seguir, retorna 'para #1 -- #2 até #3 faça'
\algnewcommand\algorithmicto{\textbf{até}}
\algrenewtext{For}[3]%
{\algorithmicfor\ #1 $\gets$ #2 \algorithmicto\ #3 \algorithmicdo}

\sloppy

\title{Classificação de Texto: \\ Acelerando o \knn{} com uma Estrutura de Árvore}

%Trecho tirado do e-mail recebido pela organização:

%Todos os trabalhos devem ser submetidos de forma anônima, ou seja, devem
%evitar qualquer informação no cabeçalho do artigo, no texto e nas
%referências que vinculem o artigo à identidade dos autores e suas
%instituições. Trabalhos que não cumprirem esta regra serão rejeitados.


%Devo deixar o trecho abaixo?
%\author{Lucas B. de Andrade\inst{1}, Elias de Oliveira\inst{1}}

%É o programa de pós-graduação mesmo?
%\address{Programa de Pós-Graduação em Informática
%    \\Universidade Federal do Espírito Santo (UFES)
%    \\CEP 29.075-910 - Vitória - ES - Brasil
%  \email{lucasbertandrade@gmail.com,
%            elias@lcad.inf.ufes.br }
%}

\begin{document} 

\title{Projeto Exemplo 
}

\author{\IEEEauthorblockN{Lucas Andrade}
\IEEEauthorblockA{\textit{Departamento de Informática} \\
\textit{UFES}\\
Vitória \\
lucasbertandrade@gmail.com}
\and
\IEEEauthorblockN{Elias Oliveira}
\IEEEauthorblockA{\textit{Departamento de} \\
\textit{UFES}\\
Vitória \\
email address}
\and
\IEEEauthorblockN{Matheus Nogueira}
\IEEEauthorblockA{\textit{Departamento de Informática} \\
\textit{UFES}\\
Vitória \\
email address}
\and
\IEEEauthorblockN{Marcos Spalenza}
\IEEEauthorblockA{\textit{Departamento de Informática} \\
\textit{UFES}\\
Vitória \\
email address}
}

\maketitle

%\input{00.1-abstract.tex}
     
\begin{abstract}
  Reduzir o esforço de clusterização através da ponderação por Algoritimos genéticos.
\end{abstract}

\begin{IEEEkeywords}
inteligência artificial, classificação automática, estrutura de dados
\end{IEEEkeywords}

\section{Problema}

\label{secintro}

O ex-presidente Michel Temer desembarcou no aeroporto de Congonhas, na Zona Sul de São Paulo, na noite desta segunda-feira (25). Ele deixou a superintendência da Polícia Federal no Rio de Janeiro, local onde estava preso desde a última quinta-feira (21), no final desta tarde.

Temer chegou em sua casa, na Zona Oeste da cidade, pouco antes das 22h.

Ser ou não ser, eis a questão: será mais nobre
Em nosso espírito sofrer pedras e flechas
Com que a Fortuna, enfurecida, nos alveja,
Ou insurgir-nos contra um mar de provocações
E em luta pôr-lhes fim? Morrer.. dormir: não mais.
Dizer que rematamos com um sono a angústia
E as mil pelejas naturais-herança do homem:
Morrer para dormir… é uma consumação
Que bem merece e desejamos com fervor.
Dormir… Talvez sonhar: eis onde surge o obstáculo:
Pois quando livres do tumulto da existência,
No repouso da morte o sonho que tenhamos
Devem fazer-nos hesitar: eis a suspeita
Que impõe tão longa vida aos nossos infortúnios.
Quem sofreria os relhos e a irrisão do mundo,
O agravo do opressor, a afronta do orgulhoso,
Toda a lancinação do mal-prezado amor,
A insolência oficial, as dilações da lei,
Os doestos que dos nulos têm de suportar
O mérito paciente, quem o sofreria,
Quando alcançasse a mais perfeita quitação
Com a ponta de um punhal? Quem levaria fardos,

O ex-presidente foi solto após decisão do desembargador Antonio Ivan Athié, do Tribunal Regional Federal da 2ª Região.

Ele permaneceu preso por quatro noites em uma sala da corregedoria, no terceiro andar do prédio da PF. O local, com cerca de 20 m², é uma das poucas salas no edifício com banheiro privativo. O espaço tinha também frigobar e ar-condicionado, além da previsão da instalação de uma TV.





\section{Solução}

Gemendo e suando sob a vida fatigante,
Se o receio de alguma coisa após a morte,
–Essa região desconhecida cujas raias
Jamais viajante algum atravessou de volta –
Não nos pusesse a voar para outros, não sabidos?
O pensamento assim nos acovarda, e assim
É que se cobre a tez normal da decisão
Com o tom pálido e enfermo da melancolia;
E desde que nos prendam tais cogitações,
Empresas de alto escopo e que bem alto planam
Desviam-se de rumo e cessam até mesmo
De se chamar ação.


Após fazer exame de corpo de delito, Temer saiu da superintendência por volta das 18h40. Alguns manifestantes estavam no local e protestaram.

A prisão de Michel Temer aconteceu em São Paulo e logo depois o ex-presidente foi transferido para o Rio.

\section{Resultados}

\input{03-classificacao.tex}

%\section{O Método}

%\input{04-metodo.tex}

%\section{Experimentos}

%\input{05-experimentos.tex}

%\section{Conclusão}

%\input{06-conclusao.tex}

\bibliographystyle{IEEEtran}
\bibliography{07-referencias}

\end{document}
