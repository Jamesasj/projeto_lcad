\label{secintro}

O ex-presidente Michel Temer desembarcou no aeroporto de Congonhas, na Zona Sul de São Paulo, na noite desta segunda-feira (25). Ele deixou a superintendência da Polícia Federal no Rio de Janeiro, local onde estava preso desde a última quinta-feira (21), no final desta tarde.

Temer chegou em sua casa, na Zona Oeste da cidade, pouco antes das 22h.

Ser ou não ser, eis a questão: será mais nobre
Em nosso espírito sofrer pedras e flechas
Com que a Fortuna, enfurecida, nos alveja,
Ou insurgir-nos contra um mar de provocações
E em luta pôr-lhes fim? Morrer.. dormir: não mais.
Dizer que rematamos com um sono a angústia
E as mil pelejas naturais-herança do homem:
Morrer para dormir… é uma consumação
Que bem merece e desejamos com fervor.
Dormir… Talvez sonhar: eis onde surge o obstáculo:
Pois quando livres do tumulto da existência,
No repouso da morte o sonho que tenhamos
Devem fazer-nos hesitar: eis a suspeita
Que impõe tão longa vida aos nossos infortúnios.
Quem sofreria os relhos e a irrisão do mundo,
O agravo do opressor, a afronta do orgulhoso,
Toda a lancinação do mal-prezado amor,
A insolência oficial, as dilações da lei,
Os doestos que dos nulos têm de suportar
O mérito paciente, quem o sofreria,
Quando alcançasse a mais perfeita quitação
Com a ponta de um punhal? Quem levaria fardos,

O ex-presidente foi solto após decisão do desembargador Antonio Ivan Athié, do Tribunal Regional Federal da 2ª Região.

Ele permaneceu preso por quatro noites em uma sala da corregedoria, no terceiro andar do prédio da PF. O local, com cerca de 20 m², é uma das poucas salas no edifício com banheiro privativo. O espaço tinha também frigobar e ar-condicionado, além da previsão da instalação de uma TV.



