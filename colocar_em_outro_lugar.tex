Um professor passa por essa mesma situação: há atividades e provas (dados) de diversos alunos, as quais ele precisa atribuir uma nota (classificar). Esse caso é mais extremo em turmas de ensino a distância, como no Ambiente Virtual de Aprendizagem (AVA) Moodle, onde pode haver um número massivo de alunos pelos quais um professor é responsável. O uso de uma ferramenta capaz de realizar esse processo de forma automática iria auxiliar e acelerar consideravelmente o trabalho do professor. Em turmas de Programação, por exemplo, os códigos-fonte de atividade submetidas por alunos poderiam ser classificados quanto a sua qualidade, e comparados entre si verificando a existência de plágio \cite{Campana:2016}, para que assim uma nota seja estimada automaticamente, poupando o árduo trabalho de correção do professor.

Já em navegações na Web, usuários se veem inundados com tamanho conteúdo disponível para consumo, e dos mais diversos tipos, como filmes, músicas, jogos etc. Isso também é um problema para os produtores de conteúdo, que precisam saber que tipo de conteúdo está em alta, e o perfil dos usuários que o consomem. Torna-se imprescindível, então, tanto para os usuários, quanto para os produtores de conteúdo, que sejam desenvolvidas maneiras eficientes de classificar os tipos de usuários e os tipos de conteúdo, para que cada pessoa receba o que é de seu interesse.